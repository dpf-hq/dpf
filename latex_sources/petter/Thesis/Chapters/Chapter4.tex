% Chapter Template

\chapter{Problem Specification} % Main chapter title

\label{Chapter4} % Change X to a consecutive number; for referencing this chapter elsewhere, use \ref{ChapterX}

\lhead{Chapter 4. \emph{Problem Specification}} % Change X to a consecutive
% number; this is for the header on each page - perhaps a shortened title

%----------------------------------------------------------------------------------------
%	SECTION 1
%----------------------------------------------------------------------------------------

\section{Problem Specification}
\label{problem}

The DPF Workbench already includes support for model transformations. However,
the framework does not have support for transforming a model between different
modeling langauges. One of DPF's strengths is that it is possible to formally
define a Domain Specific Modeling Language by defining multiple levels of
meta-models. What we want to do is to include tool support for the DPF Workbench
that change a model from one meta-modeling hierarchy to another meta-modeling
hierarchy regardless of the source models abstraction layer. This leads to the
main question for this thesis.

\begin{itemize}
  
  \item Can we include tool support for model to model transformations for the
  DPF Workbench that translates a model specified over a modeling hierarchy to a
  model specified over another modeling hierarchy?

\end{itemize}

The solution to this is not written in stone, and there are several approaches
to how we can solve this specific problem. The tool requires some implementation
but there are several existing approaches available that provides model
transformations as we mentioned in section~\ref{tooling}. DPF specifications are
basically graphs, more specific, they are directed graphs. This means that a
DPF specification consist of a set of nodes, arrows and two functions that
preserves the source and target node. This makes a graph based approach to model
transformations convenient, but we should also consider other approaches to
model transformations. 

\begin{itemize}
  
  \item Can we integrate an existing model transformation environment with the
  Diagram Predicate Framework Workbench?

\end{itemize}

We want to introduce the DPF Workbench with tool support that includes model
transformations. This has already been successfully introduced to the workbench
environment in Anders Sandven\cite{Sandven_thesis}'s master thesis. In his
thesis he describe how he integrated a M2T transformation environment to the DPF
Workbench. He integrated a model transformation environment, Xpand\cite{Xpand}
that provides a template based approach to Model2Text transformation. For this
thesis however, we want to verify that we can successfully introduce a model
transformation environment that supports translation between different DSML's.
But first we have to find a applicable environment that can be integrated with
the DPF Workbench. In section~\ref{tools} we will explore three different
model transformation tools that supports both exogenous and endogenous model
transformations. One aspect of model transformations that is required to
translate specifications in DPF is a set of transformation rules that describes
how a target model is produced. This leads to a problem for the DPF, because a
transformation rule requires modeling elements from some abstract syntax to
specify a structural pattern that is used to locate matches in a source model. 

\begin{itemize}  

  \item How can we include the abstract syntax of a modeling language that is
  specified to a corresponding linguistic meta-model and an corresponding 
  ontological meta-model for a single transformation rule.

\end{itemize}

In 2007 Ralf Gitzel, Ingo Ott and Martin Schader published a paper where they
amongst other subjects discuss the difference between Linguistic and
Ontological meta-modeling. They provide a definition between the two, 
\textit{``Linguistic metamodeling uses a metamodel to describe a language syntax
without a concrete real-world mapping. Ontological metamodeling uses metamodels
to describe domain-specific hierarchies"}\cite{gitzel2007ontological}. As it is,
the MOF 2.4.1 standard does not allow for more than a four layered
meta-modeling. DPF has an Ecore specified meta-model that describes the
language syntax and potentially unlimited layers of meta-models that describes the
domain-specific hierarchy. Figure~\ref{fig:core_metamodel} gives a
representation on how specifications are related and regardless of abstraction
layer every specification $\spec{S}$\textsubscript{1\ldots n} conforms to one
common meta-model.

\begin{figure}[H]
	\centering
	\includegraphics[scale=0.7]{./Figures/metamodelSpecification_1.png}
	\caption[Specification relationship with core meta-model]
	{Relationship between layers of specification.}
	\label{fig:core_metamodel}
\end{figure}

This model is the linguistic meta-model that DPF provides to describe the
abstract syntax for every single specification created by the DPF Model Editor.
However, other than consisting of an underlying graph and a set of constraints,
a DPF specification $\spec{S}$\textsubscript{n} is also an instance of another
specification$\spec{S}$\textsubscript{n+1}, and this is where it gets
challenging. Because in the DPF Workbench a specification model
$\spec{S}$\textsubscript{n} is created as an instance from a specification model
$\spec{S}$\textsubscript{n+1}. We can describe these specification models as
ontological meta-models, since these models describes a domain specific modeling
language through an arbitrary hierarchy of models. This is where we have to find
a work around for our solution, because model transformation environments that
utilize Ecore based models does not allow Ecore instance models to represent
abstract syntax. This could serve a potential problem when integrating the model
transformation environment with DPF. Modeling elements that DPF provides are
created as nodes and arrows from an Ecore based meta-model, but are at the same
time created according to modeling elements one abstraction layer higher.

We will discuss how we address and approach these problems in the next
chapter. But first we will look at some related work to model transformations.
We have considered the tools, The Attributed Graph Grammar
System\cite{Taentzer2004} (AGG) and Henshin\cite{Henshin_2010} that provides a
graph based approach to model transformation. We have also worked with ATLAS
Transformation Language\cite{ATL_USERMAN} (ATL), that provides a mixture of
model transformation techniques and is therefore often referred to as a hybrid
approach to model transformation. Through working with these three tools we can
find a model transformation environment that is best suited to be integrated
with the DPF Workbench.

\section{Three different model transformation environments}
\label{tools}

These model transformation tools use different approaches to how model
transformations are applied. We have looked at tools that implement classical
rewriting steps that utilizes the theory behind graph transformations and tools
that does this differently. For this survey we have tackled a specific
exogenous model transformation example, that translates an instance model of
UML's activity diagram to an instance model of a Petri
Nets\cite{jensen2007coloured} model. The next two figures provides the abstract
syntax of the two corresponding languages. These figures are represented as
Ecore models, that is EMF's interpretation of OMG's MOF. It is convenient to
represent these meta-model as Ecore models since both Henshin and ATL specify
transformation rules according to such models. First we will quickly describe
the corresponding abstract syntax for the two models before we consider the
first model transformation tool.

\begin{figure}[H]
	\centering
	\includegraphics[scale=0.7]{./Figures/ActivityMetamodel.png}
	\caption[Abstract syntax of the source model]
	{Abstract syntax of the source model for this test case.}
	\label{fig:activity_metamodel}
\end{figure}

The abstract syntax for the source model has an arbitrary number of
activities and next elements. Figure~\ref{fig:activity_metamodel} describes that an activity
element can have a name and a kind. The next element can have an inscription
and provides the property to either begin or end activities. The collection of
activities and next elements are provided by a specific activity diagram that.

\begin{figure}[H]
	\centering
	\includegraphics[scale=0.7]{./Figures/PetriNetsMetamodel.png}
	\caption[Abstract syntax of the target model]
	{Abstract syntax of the target model for this test case.}
	\label{fig:petrinet_metamodel}
\end{figure}

The abstract syntax for the target model consist of places and transitions. A
Petri net instance must have a place connected to a transition or the other way
around, but a Petri net model can never have two of the same types connected.
For this test case we defined two nodes that specify if a connection is between
a place and a transition or a transition and a place. Note that these two
meta-models are a simplified version of the abstract syntax. For this test case
we are more concerned with how the different model transformation environments
refers to the abstract syntax for the transformation rules. For each tool, if it
is either a graphical editor or a textual editor, we discuss how to edit
transformation rules, which is relevant for how we want to design a
transformation tool for the DPF Workbench. We then consider how the different
tools defines the abstract syntax for both the source and target model. Next we
specify how transformation rules are created and if the tool provides any
application control for these rules. We finish up by mentioning how the tool
applies this model transformation. In section~\ref{tool_choice} we consider the
model transformation environment that can be integrated and provides a viable
model transformation technology that works with the DPF Workbench.

\subsection{The Attributed Graph Grammar System}

AGG is a general development environment for algebraic graph
transformation systems that provide a graphical editor for creating
and modifying graphs. The editor provides a graphical user-interface with
several visual editors for applying the principles of graph transformations. AGG
also provide an interpreter and a set of validation tools. The system is an
ongoing research activity of the graph grammar group at TU Berlin and started
in 1997.

\subsubsection*{Graphical Editor}
The graphical editor of AGG, represented in figure~\ref{fig:AGGScreen}, has
several functions to help the user to define model transformations. In the top
left corner of the graphical user-interface is a tree based editor that provides
a set of transformation rules, type graphs, and host graphs. The host graph
represents both the source and target model in a model transformation, and the
type graph represents the abstract syntax for the modeling languages. The
source-target relationship of the host graph is one and the same, but we will
discuss this in future sections, but for the purpose of this thesis we will
refer to the host graph as the source graph.

Each transformation rule has two visual editors, representing the left
(LHS) and the right hand side (RHS), or the pattern and the replacement graph.
In the tree based editor it is also possible to attach application conditions to
transformation rules. This is convenient if the user wants to specify
constraints that restricts the pattern or replacement graph to be applied
accordingly to the specific application condition.

Type graphs are described more in depths in the next section, but roughly said,
the type graph defines modeling elements that can be used to create the source
graph and is similar to how Ecore defines meta-models for EMF. The users can
now create model instances that represents the concrete syntax of a specific
modeling language. This representation of the abstract syntax are represented
in a source and corresponds to its concurrent type graph.

The transformation rules can be extended with Java expressions. This means that
the users can use Java primitives such as strings, integers or float numbers to
form the pattern graph or the left hand side of the rule. However, the users
can only bind attributes that are in a corresponding type graph.

Figure~\ref{fig:AGGScreen} also represents some node elements and association
elements. These are meta elements that are initialized in the type graph and
are used to model the source graph, the different transformation rules and
application conditions. The node elements and association elements also
describes the semantics of the type graph. Note that both the node elements and
association elements in the figure has been scaled up for the purpose of this
paper.

\begin{figure}[H]
  \centering
    \includegraphics[scale=0.3]{figures/AGGscreen.png}
    \rule{35em}{0.5pt}
  \caption[Graphical Editor for AGG]
  {A model transformation for the AGG Editor.}
  \label{fig:AGGScreen}
\end{figure}

\subsubsection*{Defining Meta-models}

Before the source graphs can be created, we have to specify the modeling
language for the source model and the target model. In AGG both the source and the
target meta-model are defined in one common type graph. The type graph
represents the abstract syntax for both the source and target model. If we want
to prepare an AGG graph for a model transformation, we create a type graph with
references between source modeling elements and target modeling elements. AGG is
unaware of the relationship between these modeling elements unless we explicitly
initialize them. The relationship between source and target modeling elements in
the abstract syntax has two major purposes for an exogenous model
transformation. The relationship specifies how source modeling elements
correspond to target modeling elements and determines upon execution of
transformation rules that a matched pattern is only found once. 

\begin{figure}[H]
	\centering
	\includegraphics[scale=0.7]{figures/AggTypeGraph.png}
	\rule{35em}{0.5pt}
	\caption[Type graph in AGG]
	{Type graph for activity diagram and Petri Net in AGG.}
	\label{fig:AggTypeGraph}
\end{figure}

Figure~\ref{fig:AggTypeGraph} represents the type graph for our test case. The
abstract syntax contains nodes and arrows that include a structural multiplicity
constraint. The user defines nodes and arrows for each meta element for both
the source and target model. This is achieved by using either the Edge Type
Editor or Node Type Editor and are editors that correspond to either a node or
an edge. The nodes and edges are given names and graphical properties such as
colors or shapes. Nodes represents modeling elements from the two modeling
languages while arrows represents the associations between these modeling
elements. In the type graph we want to distinguish between associations and
correspondences, and therefore we represent a correspondence between a source
and target modeling element as a dashed arrow. The dashed arrow has the same
properties as the association arrow between nodes, but the graphical
representation is different. This makes the concrete syntax in the source graph
easier to read when we are applying the transformation rules. In
figure~\ref{fig:AggTypeGraph} we can see that a RefAct node is defined and is
connected between the activity element and the transition element. The same
initialisation is defined between the next element and the place element. This
reference edge specifies that there is a correspondence between Activity and
Transition element and between Next and Place. For this type graphs there is a
structural multiplicity constraint for the nodes and edges. This means that
there can be an arbitrary, or a zero to many number of instances of these nodes
and edges in the source graph.

\subsubsection*{Defining Transformation Rules}
\label{sec:AGG_rules}
Now the type graph has been initialised and the instance graph of the
source model has been created. But to be able to translate to a target model,
we need to create a set of transformation rules. A transformation rule is
defined with an unique name, an empty LHS graph and an empty RHS graph. 

\begin{figure}[H]
	\centering
	\includegraphics[scale=0.7]{figures/AGGTreeBasedEditor.png}
	\rule{35em}{0.5pt}
	\caption[Tree based editor in AGG]
	{Tree based editor for transformation rules.}
	\label{fig:AGGTreeBasedEditor}
\end{figure}

Whenever changes are made in the two graphs, AGG checks if the LHS or the RHS
conforms to the type graph. The user is unable to insert elements in the two
graphs that are not initialised in the type graph and the users are not allowed
by AGG to create associations between nodes that are not initialised in the type
graph. This is how AGG keep the source and target models consistent. In
figure~\ref{fig:AGGTreeBasedEditor} we can see the tree based editor in AGG,
that provides the type graph, the source graph and a list of transformation
rules. When a new rule is created, both the LHS and the RHS are initialised. The
users can then specify a graph structure that forms the LSH graph that AGG use
to locate matching patterns in a source graph. The RHS graph represents the
graph structure that the transformation system produces for each located match.
AGG provides two visual editors for these corresponding graphs. However, there
is also a graph that represents the intersection between the LHS and the RHS. In
AGG this graph is edited by creating similar modeling elements in both the LHS
and RHS graph and map these modeling elements with each other. 


\begin{figure}[H]
	\centering
	\includegraphics[scale=0.5]{figures/LHSvsRHSAGG.png}
	\rule{35em}{0.5pt}
	\caption[Representation of a rule in AGG]
	{The LHS and RHS of a rule and a NAC attached.}
	\label{fig:LHSvsRHSAGG}
\end{figure}

Figure~\ref{fig:LHSvsRHSAGG} is a representation of the rule
transformNextToSimple, with the LHS and the RHS graph. The LHS contains the
graph structure that is used to locate matches while the RHS contains the
graph structure that replaces the located matching pattern. For this rule the
modeling elements that are represented in the LHS are also part of the graph
structure in the RHS graph. Modeling elements that are part of both the LHS and
RHS graph can be mapped together to specify for a transformation engine that
these modeling elements defines a graph structure that is an intersection
between the LHS and the RHS graph. In section~\ref{sec:graph_based} we
introduced the concepts of single and double pushout of graphs for graph based
model transformation tools. A double pushout of graphs includes this
intersection graph for a transformation rule that makes it possible to preserve
modeling elements that is part of both the LHS and the RHS. Modeling elements
that is defined in the LHS graph and not the RHS graph are removed when the AGG
transformation engine applies a transformation rule. For example the
transformation rule in figure~\ref{fig:LHSvsRHSAGG} locates the graph structure
that the LHS defines in a source graph and inserts a new graph structure that
the RHS defines in the source graph. This transformation rule will not remove
any modeling elements once applied since we have specified that the modeling
elements that is part of the LHS graph are also part of the RHS graph. A rule
can also specify application conditions that can either be a Positive
Application Condition (PAC) or a Negative Application Condition (NAC).
Figure~\ref{fig:LHSvsRHSAGG} has a NAC, activitySimple that makes sure that the
LHS of the rule is translated only once for each pattern match found in the
source graph. This is because for each applied transformation rule we preserve
the LHS graph structure and therefore is a potential match the next time the
transformation rule is applied. However, the negative application condition
requires that the matching pattern should not contain any references and
therefore is not a valid matching pattern. Through the use of these application
conditions, the users can create restrictions to how each transformation rule
should handle matching patterns in the source graph. A transformation rule
can have multiple application conditions attached.

\subsubsection*{Application Control for the Transformation Rules}

In subsection~\ref{sec:app_control} we specified that the application control of
a model transformation handles the location of matches and rule application
control. Locating matches in AGG are designed by some non-deterministic search
algorithms that the users have no control over. AGG does however provide the
users with the possibility to explicitly specify how the transformation rules
are applied. By default, the transformation rules are applied
non-deterministically. This means that there are no pattern to how the
transformation rules are applied and the transformation rules can be applied
differently on different runs. This option is quite useful if the set of
transformation rules are independent of each other. AGG also provides other
ways of applying rules such as applying rules by layers or by sequences. When
the transformation is set to be applied by rule layers, then AGG introduce
the users with an integer that specify transformation rules on different layers.
This layer number will range from 0 \ldots n, where the lowest number is the
first layer and therefore the has first priority. If there are rules with
the same layer number, then these rules will be internally applied
non-deterministically. If the rules are applied by sequence then the rules will
be applied from the first element in the tree based editor and applying the rest
of the rules in sequence.

\subsubsection*{Translating the Source Graph}

In section~\ref{sec:AGG_rules} we described that when applying a transformation
rule a matching pattern is located in the source graph and a graph structure
from the RHS is inserted in the source graph. This is special source models in
AGG, because the source graph in AGG represents both the source model and the
target model in a model transformation. AGG's transformation system provides an
in-place model transformation directly on the source graph. An exogenous model
transformation that is specified in AGG usually include all located matches in
the translated graph structure for each transformation rules. These modeling
elements that represents the abstract syntax of the source model can then be
removed through a set of transformation rules after the modeling elements
that correspond to the abstract syntax of the target model is translated. This
means that for an exogenous model transformation we should be careful when
applying the transformation rules non-deterministically. The user can now
either press Start Transformation or do the transformation one step at the
time. For the first option AGG will apply one rule at the time until there are
no more matching patterns located in the source graph. When AGG cannot find
any more matches, the host graph is either correctly translated or there are
errors in the rules. The user can also execute the transformation step by step.
This will give the user the same result as the first option, but now the user
can do one match at the time for each rule. AGG utilize both the single and
double pushout approach when executing a transformation
rule\cite{Taentzer2004}. Like we discussed in section~\ref{sec:graph_based} the
single pushout approach removes the graph structure from the LHS and inserts the
graph structure from the RHS in the source graph. If the rules specifies an
intersection graph between the LHS and RHS graph then the double pushout
technique is used. AGG's transformation engine interpret a transformation rule
and applies the transformation rule accordingly. Another model transformation
environment that similar to AGG utilizes the concepts of graph transformations
is Henshin.

\subsection{The Henshin Project}

The Henshin project\cite{Henshin} provides a transformation
language and a tool environment for defining model transformations for the
Eclipse Modeling Framework. The Henshin project is part of the Eclipse Model
Framework Technology (EMFT). EMFT acts as an EMF subproject for new
technologies that extends and utilize EMF. The Henshin Editor was initially
developed in a student project at Technical University of Berlin in 2010, and
extended in the bachelor thesis \cite{JohannSchmidt} published by Johann Schmidt
and the master thesis \cite{AngelineWarning} published by Angeline Warning.
The Henshin project provides a transformation language based on graph
transformations that supports both endogenous and exogenous model
transformations. With the help of a graphical editor, Henshin provides the user
with an intuitive approach to defining transformation rules. The Henshin tool
environment provides a transformation engine, several editors and a state space
generator.

\subsubsection*{Graphical Editor}
Henshin model transformation environment is integrated as a plugin for the
Eclipse Integrated Development Environment\cite{Eclipse} and provides a
graphical editor to create and modify transformation rules. 

The users start out with using the Eclipse wizard to create an empty Henshin
document. The Henshin document is based on the commonly known Extensible Markup
Language (XML)\cite{XML}. If applicable a Henshin diagram file can be created
based on the Henshin file that gives the users an intuitive approach to
creating transformation rules.

The Henshin transformation file is represented in a tree based editor called
the Henshin Model Editor. Figure~\ref{fig:Henshin_TreeEditor} represents the
editor that contains a list of transformation rules. These transformation rules
are included under a Module element that represents the root element for a
Henshin model transformation. For this specific example there are two external
Ecore models included in the editor, more specifically the source and target
meta-models. These meta-models are created based on the EMF standard for
creating models and are independent of each other. Please note that a Henshin
model transformation can include 0 \ldots n models and therefore is not
restricted to have exact one source and one target meta-model.

\begin{figure}[H]
	\centering
	\includegraphics[scale=0.7]{figures/Henshin_TreeEdtiro.png}
	\rule{35em}{0.5pt}
	\caption[The Henshin Model Editor]
	{Tree Based Editor for rules in Henshin.}
	\label{fig:Henshin_TreeEditor}
\end{figure}

\subsubsection*{Defining Meta-models}

The Henshin language requires a source and a target meta-model to be able to
perform model transformations. The target meta-model can either be the same as
the source meta-model or defined in another modeling language. Either way, before
the users can start creating transformation rules, the meta-models has
to be defined. To define these meta-models, Henshin utilizes Ecore, that is
provided by the Eclipse Modeling Framework\cite{Steinberg2009}. Ecore models
can either be created using a tree based editor, called Sample Ecore Model
Editor or by using a graphical editor. While the graphical editor is optional,
the tree based editor is mandatory for creating Ecore models, since this
represents the Ecore model.

Initially the user has to create an EPackage element in the newly created Ecore
model. Henshin interpret instances of this EPackage as EObjects and is what
Henshin searches for when the user want to import an Ecore model. This EPackage
element can have several child elements, like for example EClass, EEnum and
EData type. For this specific example we only needed the EClass element to
create the nodes for the meta-models. For each EClass element the users can
specify an EReference that connects two EClass elements. This means that an
EReference element defines relations between the nodes for the meta-models. To
give an EClass element properties, the user can create an EAttribute element.
This element can be typed, either by a predefined list of types or by defining
user created EData types. For the purpose of this case study we only needed to
name the different nodes and therefore we only needed the data type EString.
Through the use of these Ecore elements, we can create the two meta-models from
figure~\ref{fig:activity_metamodel} and figure~\ref{fig:petrinet_metamodel} that
was previously presented in this chapter. 


\subsubsection*{Defining Transformation Rules}

Now we have defined the source and target meta-model, and imported
both of the meta-models EPackages. We can now use elements from the two
meta-models to create transformation rules in the Henshin transformation
language. In Henshin, objects are referred to as nodes and links between objects
as edges. From the meta-models these nodes represents the EClass elements and
edges is a EReference between these EClass elements. A collection of these
nodes and edges defines a graph structure. Each transformation rule in Henshin
specifies two graphs that represent the LHS and the RHS. Note that the
graphical editor provides an integrated view to creating transformation. And
therefore Henshin handles assignment of modeling elements to the LHS and the
RHS through the use of stereotypes. Figure~\ref{fig:HenshinScreen} represents a
visualization of the graphical syntax and includes a transformation rule,
``transformSimpleActivity''. On the right side there is a palette that contains
Henshin modeling elements and different EPackages. The first two EPackages
represents modeling elements for the source and target meta-model. The Henshin
Trace Model provides support for including traceable links for exogenous model
transformations in Henshin. The Henshin Trace model provides a traceable link
that keeps track of the translated elements during a transformation. This model
consist of a single class Trace, that has two references called source and
target. These references are of type EObject and therefore can refer to any EMF
object. The Trace model is generic and therefore supports creation of traceable
links between any Ecore models.

\begin{figure}[H]
	\centering
	\includegraphics[scale=0.6]{figures/henshin_scren_2}
	\rule{35em}{0.5pt}
	\caption[The Henshin graphical editor]
	{A rules represented in the Henshin graphical editor.}
	\label{fig:HenshinScreen}
\end{figure}


The Node, Edge and Attribute modeling elements are used to define the different
transformation rules in Henshin. A new transformation rule in Henshin always
have to start with creation of a new Rule element. Inside this Rule element the
users are free to create nodes, and connect these nodes with edges. The nodes
and edges defines a graph structure that is used to either locate matches in a
source model or translate target modeling elements. Henshin makes sure that
these nodes and edges conforms to their corresponding meta-models. Note that the
Node and Edge modeling elements are special for Henshin and are used to create
the content of the transformation rules. The Node element has a type that
correspond to an EClass while the Edge element has a type that correspond to an
EReference. Note that the Ecore models that are represented in
figure~\ref{fig:HenshinScreen} has a list of modeling elements that are typed by
EClass. These modeling elements are a shortcut for Henshin to create nodes and
creates a Henshin node with the corresponding type. The Attribute element can be
used if attributes are defined for the classes that are imported. We will come
back to the Unit element in the next section. Henshin distinguish if nodes,
edges or attributes are part of the LHS and the RHS through the use of
predefined stereotypes, or action types. Based on these action types Henshin
automatically specifies if these modeling elements defines a graph structure
that is used to locate matches in a source model or produce modeling elements
for a target model. If the action type consist of the sequence ``create'',
Henshin knows that this element should be part of the replacement graph, or the
RHS. While on the other side, the sequence ``delete'' should be part of the
pattern graph, or the LHS. The ``preserve'' sequence is a bit more special,
because nodes or edges in Henshin that is specified by this action type should
be part of both the LHS graph and the RHS graph. This is done by putting the
preserve element in both graphs and then create a mapping between these two
elements to inform the Henshin Interpreter that this represents the same
element. Henshin also has support for application conditions. The action types
``forbid'' and ``require'' are used for defining Negative Application
Conditions (NACs) and Positive Application Conditions (PACs). These actions are
supported for nodes, edges and attributes. The example rule in
figure~\ref{fig:HenshinScreen} use four of these action types. The modeling
elements in gray represents modeling elements that are part of both the RHS and
the LHS graph, while the modeling elements in green are specified in the RHS
graph. This specific transformation rule will locate a matching pattern in a
source model that is described by an Activity modeling element. The positive
application condition specifies that Henshin only should locate matches that is
of kind ``simple''. The negative application condition specifies that a
located match that is described by the Activity class should not have a
traceable link. The NAC specifies that the transformation engine does not locate
duplicate matching patterns. The first time a match is located a traceable link
is established. Now this specific match is no more a valid match since the NAC
forbids the transformation engine to locate matches that already has established
a traceable link to this modeling element. Note that we have a Trace and a
PetriNets that is both a preserve modeling element. This is because these two
are already translated by another transformation rule.

\subsubsection*{Application Control for the Transformation Rules}
Transformation units are used to administrate the different transformation
rules. Henshin provides several different units with different properties.
Note that a transformation rule is also a transformation unit. This means that
it is unnecessary to create a unit in Henshin if our model transformation only
consist of one transformation rule. But if there are more than one 
transformation rule there has to be a control mechanism that determines how
these transformation rules should be applied. An Independent Unit is applies
rules non-deterministically and is a good solution if the order of applying the
transformation rules is not important. But if the transformation rules requires
a very strict pattern and are dependent of other rules, then a sequential unit
are a safe way to apply rules. The sequential unit forces the Henshin
transformation engine to apply rules in a sequenital order.
Figure~\ref{fig:SequentialUnitHenshin} is an example of a sequential unit
that will start applying rules at the black circle and follow the arrow
through each given rule until it is finished.  

\begin{figure}[H]
	\centering
	\includegraphics[scale=0.7]{figures/SequentialUnitHenshin_1.png}
	\rule{35em}{0.5pt}
	\caption[A Sequential Unit in Henshin]
	{A SequentialUnit main that contains a sequence of rules.}
	\label{fig:SequentialUnitHenshin}
\end{figure}

If applicable a transformation unit can also consist of other units, for example
if the user want to either iterate or loop through a transformation rules. The
previous section described an example of a rule in
figure~\ref{fig:HenshinScreen}. The sequential unit above the
``transformSimple'' represents a LoopUnit while the ``transformStart'' and
``transformEnd'' represents a single rule. The loop unit applies a single
transformation rule until there are no more matches found in a source model.
This is convenient for the rule, ``transformSimpleActivity'' since we specified
that the rule should only locate matches where there exist no traceable link.
Henshin also has two other units that can administrate transformation rules,
namely ConditionalUnit and PriorityUnit. The ConditionalUnit follows a if-else
pattern, and is used if the user want Henshin to choose between other units.

\subsubsection*{Translating the instance model}

Now that we have defined the source and target meta-model, created a set of
transformation rules and initialized a control mechanism for these rules it is
time to apply the transformation rules. For Henshin there is two ways to do
this. In Henshin the default engine for executing model transformation is the
Henshin interpreter. This interpreter can be invoked either by using the a
Eclipse wizard or programmatically using the Henshin API. 

Using the Eclipse wizard is done by opening the Henshin file in the Henshin
Model Editor and right clicking the root object and locate apply transformation.
This will open a wizard where the user can choose a transformation unit. This
will either be a single transformation rule or some transformation unit that
applies all other units and rules. The user also has to select the instance
model and can explicitly set parameters for the rules if this is applicable. If
the parameter is set to Ignore then the interpreter will automatically match the
parameter. Now the user has two choices, the first choice is to preview the
result of the model transformation. This will either show the user a new window
with the modifications to the model or a message that the rule or unit could not
be applied. If the user press Transform instead of Preview, the model will be
transformed and saved.

The interpreter can also be invoked programmatically, either as an
Eclipse based application or as a simple Java application. Henshin provides a
API that lets the users invoke the interpreter through the use of Java code.
There is a class HenshinResourceSet that lets the user load and save models and
transformations. When the instance model and Henshin module is loaded into the
resource set, the transformation can be applied through the use of the Henshin
Engine class. This is where Henshin finds and translates matches found in the
instance graph. The user also has to specify the main transformation unit from
the Henshin module. Both the engine and the unit can be loaded into the
UnitApplication class. And this class has a method called execute that lets the
user execute the model transformation. If the transformation was executed
without errors, then the instance model can be saved with the translated
changes. The Henshin API lets other users use the power of Henshin in their own
program.

\subsection{ATL Transformation Language}

ATL\cite{ATL} (ATL Transformation Language) is a model transformation
language and is an implementation of the QVT\cite{QVT} standard. It
provides ways to produce a set of target models from a set of source models.
ATL is maintained by OBEO\cite{OBEO} and AtlanMod\cite{ATLANMod} and was first
initiated by the AtlanMod team, previously called the ATLAS Group, located at
the University of Nantes in France. The initial version of ATL was created In
2004, where ATL became part of the Eclipse Generative Modeling Technologies
(GMT) \cite{GMT}. The goal of GMT is to produce a set of research tools in the
area of Model Driven Software Development. The ATL Integrated Development
Environment (IDE) was later promoted for the Eclipse M2M project in January
2007.

There are developed several tools that has support for a declarative
approach to model transformation. For the purpose of this paper, we will explain
this approach with the focus around the Atlas Transformation Language (ATL).
ATL is a hybrid model transformation approach, that is a transformation
language that combines other model to model transformation approaches. For
example, ATL provides transformation rules that can be either fully declarative
or fully imperative or a mixture of both. 

\begin{figure}[H]
	\centering
	\includegraphics[scale=0.55]{figures/ATL.png}
	\rule{35em}{0.5pt}
	\caption[Model transformation for Atlas transformation language]
	{Model transformation process for Activity2PetriNets.}
	\label{fig:ATL}
\end{figure}

Figure~\ref{fig:ATL} gives us an idea of how the ATL transformation from an activity
diagram to Petri net are handled. We want to generate a instance of PetriNets,
that conforms to its own meta-model. This is generated from a source
model, Activity Instance, that conforms to its respective meta-model.
The created transformation Activity2PetriNets is expressed in the ATL
transformation  language, that conforms to its own meta-model. These three
meta-models conform to the meta-model Ecore. So this makes Ecore a metameta-model
to represent the meta-models of Activity, ATL and PetriNets.

ATL has to be configured properly before the user can execute a model
transformation. In this configuration both the location of the source and target
meta-model has to be specified. The user also has to specify what instance model
that should be translated. And lastly the user has to create a new file that can
be specified as the target instance model for the ATL run configuration. The
user can then initiate the transformation by running this as an ATL
transformation.


\subsubsection*{Textual editor}

ATL can be compared to a programming language, because it is
basically a transformation language that provides a concrete textual syntax. ATL
is a text based transformation language, and is build around the Object
Constraint Language (OCL) \cite{OCL} with some additional predefined functions.
ATL transformations is stored in a file extension called ``.atl'' These ATL
files can contain different kind of ATL units and are defined in its own
distinct ATL file. These different ATL units are ATL modules, ATL queries and
ATL libraries. Libraries can be used to create independent ATL libraries that
can be imported to different types of ATL units. The module unit specifies the
different application rules for a  model transformation. And the Queries are
used when the users want to compute primitive values from the source models.

Now that we have specified these three ATL units, we can describe shortly how
we can use the ATL transformation language to create model to model
transformations. For our case study, we only need the ATL modules. An ATL module
corresponds to a model to model transformation. This unit enables developers to
specify the way to produce a set of target models from a set of source models.
The source and target models of an ATL module must be consistent with their
respective meta-models. 

\subsubsection*{Defining Meta-models}

Defining meta-models for the ATL language is defined by the modeling language
Ecore. Since defining the meta-models are defined similar as Henshin, see
chapter 4.2 for more details. 

At first, the user start out with a blank ATL file. Since we are working in
the ATL Integrated Development Environment for Eclipse, we want to start the
document with defining the path to the source and the target meta-model. The
reason for doing this is to achieve auto completion from elements defined in the
Ecore meta-models. This is convenient for the users when creating transformation
rules.  

\begin{figure}[H]
	\centering
	\includegraphics[scale=0.5]{figures/ATLScreen.png}
	\caption[Simple rules for ATL]
	{Two simple rules for Activity2PetriNets in ATL.}
	\label{fig:ATL_Screen}
\end{figure}

Next the file is composed of four different elements. The first element is the
header section, where the user can give the module a name and name the variables
corresponding from the source and target models. The module name has to be
identical to the name of the ATL file.

We also need to specify the source and the target meta-model. From
figure~\ref{fig:ATL_Screen} we can see that the target meta-model is initialised 
with the keyword create, and the source meta-model is initialised using the
keyword from. The user can also import some existing libraries if needed. This
import section is however optional. Importing meta-models are handled a bit
differently in ATL compared to Henshin. In ATL the meta-models are imported
explicitly while in Henhsin they are imported implicitly before they can be used
in modifying the transformation rules. For ATL the user has to configure where
both the source and the target meta-model are located through a configuration
page. 

The next element is a set of rules that defines how the target models are
generated from the source models. These rules are used to implicitly match
source elements and produce target elements. In figure~\ref{fig:ATL_Screen} we
have examples of two rules, namely the rule for transforming the start activity
and the rule for transforming the end activity. We can see that for each rule we
specify what we want to translate from and what we want to translate to. We will
describe transformation rules in more details in the next section.

The last element in a ATL module is a set of helper functions. This collection
of helpers can be compared to Java methods. These helper methods can be used to
make the transformation rules easier to read for example.

\subsubsection*{Transformation Rules}

A rule in ATL describes how a target model should be generated from a source
model. In ATL there are three kinds of rules, the type matched rules and the
lazy rules are both fully declarative while the called rules are imperative.
These rules has an input pattern and an output pattern. The input pattern can
have a list of source model elements that is part of a rule in ATL by defining
several input pattern elements. Each input pattern element has to have a
mandatory type that corresponds to a metaclass defined in some meta-model. Each
rule corresponding input pattern can also specify optional conditions that are
expressed as OCL expressions. Both the type and an optional condition specifies
which elements from the source model that is matched for each rule. The output
pattern defines how the target model elements are created from the input model. 

\textbf{The matched rules} provides an declarative approach to
creating transformation rules in ATL. The users can specify from which kinds of
source elements the target elements can be generated from and how the generated
target elements should be initialized. A matched rule finds a match according to
the type of source model element and generate target model elements from these
matches. A new matched rule is defined by the keyword ``rule'' and has two
mandatory and two optional sections. The mandatory sections specifies the input
pattern and the output pattern while in the first optional section the users can
declare and initialize local variables. Note that these variables can only be
used in the scope of each rule. The second optional section includes an
imperative section The type that is introduced in the input pattern conforms to
a meta-element in a meta-model of the source model. This rule will then generate
target elements according to each match in the source model.

Figure~\ref{fig:ATL_Example} shows a simple rule, Activity2StartPlace that wants
to translate Activity source elements to some target elements. This rule
specifies the keyword \textit{from} for the input pattern and \textit{to} for
the output pattern. For this example we want to find matches for one source
element that is of type Activity that conforms to the meta-model Activities. We
also provide additional properties for this input source element, where we only
want to find matches that conforms to the type Activity and has the name
``start''. The rule specifies that we want to generate three target
pattern elements p, t and a\textunderscore t from this matching type. These
generated target elements conforms to the meta-model Petrinets and specifies
that these generated types should generate attributes from the source
pattern element. The generated target model elements is initialized with
attributes from the matched source pattern element. 

\begin{figure}[H]
	\centering
	\includegraphics[scale=0.7]{figures/ATL_Example.png}
	\caption[Example of a matched rule in ATL]
	{An example of a matched rule in ATL.}
	\label{fig:ATL_Example}
\end{figure}

If applicable the users can add an optional condition for each rule to check for
certain matches for this input element. This condition is expressed as an OCL
expression and gives the user the possibility to restrict the searches of the
source elements. 

The second type ATL rules are \textbf{Lazy rules}. These lazy rules will
never be applied when a model transformation in the Atlas Transformation
Language is executed. These lazy rules can only be applied to a model
transformation when they are called from another of the two rules. These lazy
rules are created similar to the matched rules.

The third and final type for an ATL rule is called \textbf{Called rules}. A
called rule has to be called from an imperative section from either a match rule or
from another called rule. A called rule is created similar to a matched rule,
namely with a \textit{rule} keyword. One thing that is special with a called
rule is that it does not have to match source elements from the source model.

\subsubsection*{Execution of an ATL transformation}

Figure~\ref{fig:ATL_Execution} describes the architecture of the transformation
language. From the figure we can see that we have an association between EMF and
Ecore models. This are the meta-models that are expressed using EMF's Ecore
model. These meta-models are then translated through a model handler that
compiles these Ecore models to the ATL Virtual Machine. Where these meta-models
can be used both in creating ATL programs and in ATL's internal interpreter. The
ATL compiler translates the ATL file into a new ASM assembler file, that ATL can
use to launch a model transformation. This assembler file contains the
compiled code of the corresponding ATL file.

\begin{figure}[H]
	\centering
	\includegraphics[scale=0.6]{figures/ATL_Execution.png}
	\caption[Internal infrastructure for ATL]
	{Internal infrastructure of for ATL.}
	\label{fig:ATL_Execution}
\end{figure}

The default semantics for executing a set of transformation rules specified in
ATL can be described in three phases. See ATL User Manual\cite{ATL_USERMAN} for
more information.

The first phase is an initialization phase. This phase consist of amongst other
things to initialize the trace model and the module of the ATL transformation.
The trace model in ATL has one important function, and that is to create a
trace link that points to the matched source input elements and the
corresponding generated target output elements. The trace model in ATL works as
an implicit tracing mechanism that specifies relationships between the source
element and its corresponding target elements by using a native type called
ASMTransientLink\cite{Wagelaar}. For every time a transformation rule is
matched to a source element, one ASMTransientLink is created. To this transient
link the name of the transformation rule provided together with the source
element and the target elements. These links are added to a collection that is
stored internally for ATL. This means that the users of ATL cannot access these
links after a model transformation has finished executing. However, as shown by
Andr\'{e}s Yie and Dennis Wagelaar\cite{Wagelaar}, that gaining access to these
ATL traces can be done explicitly by creating transformation rules that
generates a tracing model based on the internal tracing information provided by ATL.

The next phase consist of finding matches in the source pattern of the matched
rules. This is done by the ATL transformation engine that searches for valid
matches. A match is valid when all input pattern elements are found amongst
the source model elements and any OCL expression for that matched rule is valid.
The transformation engine also allocates the target model elements based on the
declared output pattern into memory. At this point the target model elements are
only allocated, they are initialized in the final phase. For each match found,
there is created a trace link that has a source link to the matched source
elements and a target link to the generated target elements. The generated
target elements are not given any attributes or properties in this phase. This
phase create target elements from matches found, and create a trace link between
them.

The final phase of for executing an ATL module is to initialize the target model
elements. At this stage each allocated target model element are given
attributes and features that corresponds to the matched rule. The ATL
transformation engine now use the trace links to determine the matched source
elements and the generated target elements. This operation is called
resolveTemp, that returns the reference from the target model elements that
where generated in the second phase and to the corresponding source model
element. Now that these three phases is finished the ATL transformation engine
can execute the imperative code sections defined for the module. 

\section{Model transformation environment for DPF}
\label{tool_choice}

After working with the three model transformation environments in the previous
section we decided to try and integrate Henshin with DPF. In this section we
will describe why Henshin is the better choice of the three considered
environments to integrate with DPF. Henshin\cite{Henshin_2010,Henshin} is a
relatively new installment in the world of model transformations. The
environment was initially created three years ago, in 2010 and is marked as an
Eclipse Incubation project. The purpose of the incubation phase is to establish
a fully functioning open-source project. In theory an integration of Henshin
with the DPF should be possible, since Henshin applies model transformations
based on Ecore models and DPF models are basically represented as Ecore
models. This presents a problem with integrating AGG with DPF. Because AGG does
not support the Eclipse Modeling Framework. In EMF the root of all modeling
objects is an EObject. And this EObject has no references to the Java Object.
AGG could also be integrated with DPF, but the problem is that this would
require an extensive amount of manual coding. We could use AGG as a general
purpose graph transformation engine in a java applications. We would have to
create the source model as a AGG graph and a type graph based on the source and
target modeling formalism that DPF provides. AGG provides an API that
conveniently does this, but Henshin also provides an API 

Therefore AGG can not be integrated with EMF, since AGG can take Java Objects
as input, but not EObjects. Both Henshin and ATL supports EMF. This means that
the user can use the EMF API together with Henshin and ATL in a Java
Application.

AGG is a rule based visual language supporting an algebraic approach to graph
transformation. It aims at the specification and prototypical implementation of
applications with complex graph-structured data. AGG may be used (implicitly in
"code") as a general purpose graph transformation engine in high-level JAVA
applications employing graph transformation methods. Due to its rule based
character AGG may also be near in the

For Henshin we can import this meta-model, \textit{Meatmodel.ecore} and use it
to define the content of the transformation rules since every specification
$\spec{S}$\textsubscript{n} conforms to this meta-model.

The problem with integrating Henshin with DPF is that Henshin is implemented by
EMF, and therefore utilize OMG's MOF. The Henshin model transformation
language create the transformation rules based on models that are created
accordingly to the four layer meta-modeling provided by MOF. The
specification implementation for a DPF model is created as an EMF meta-model 
and utilize the EMF Genmodel to provide code generation. DPF provides
initialisation of a potential endless hierarchy of meta-models, and therefore
does not match the steps MOF provides to create abstract syntax for a domain
specific language. It is however important to separate the implementation of a
specification and the creation of a new specification. Because DPF utilize EMF
to some extent. Each model created in DPF represents the lowest layer of
meta-modeling in MOF, regardless of the level of abstraction the model
represents. However what makes DPF unique compared to EMF is that a DPF model us
an instance model of both an Ecore meta-model and another DPF model. This means
that a DPF models concrete syntax is typed by the abstract syntax of another DPF
model. While all modeling elements from these models conforms to one common
meta-model. In DPF we can create an arbitrary level of meta-models and therefore
each domain specific modeling language defined in the framework can have
different layers of specification models. The Henshin environment has strict
guidelines on how models are imported and used. These models are required to be
created accordingly to and typed by the Ecore model provided by EMF. Henshin
can then utilize these models to create a graph pattern that structure both the
LHS and the RHS graph of a transformation rule.

 because we cannot import an instance model of
an Ecore based meta-model into the Henshin model transformation environment. We
can do changes to an instance model by using Henshin, but the transformation
language can only import and utilize models that conforms to the Ecore
meta-model. To solve this for DPF models we expand transformation rules in
Henshin with application conditions. We will look more closely to how this is
done later in this chapter, but what we basically do is that we restrict the
LHS graph to locate matching modeling elements in an instance model based on
the abstract syntax that another instance model provides.

Henshin supports out of the box model transformation that translate
instance models represented at the first layer of abstraction in the MOF. These
instance models provides the concrete syntax of a modeling language and are
described by a corresponding meta-model that represents the abstract syntax.
This meta-model is provided accordingly to the second layer of the MOF. This
means that Henshin provides model transformation according to EMF's two layered
modeling environment. This is a problem for the DPF, because this framework can
specify a modeling language through an arbitrary layers of models. And we know
that a transformation rule in Henshin requires references to meta-modeling
elements from a source meta-model and a target meta-model to translate an
source model to a target model. This generates a new question.
%----------------------------------------------------------------------------------------
%	SECTION 2
%----------------------------------------------------------------------------------------

