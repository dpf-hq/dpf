% Chapter Template

\chapter{Model Driven Engeneering} % Main chapter title

\label{Chapter2} % Change X to a consecutive number; for referencing this
% chapter elsewhere, use \ref{ChapterX}

\lhead{Chapter 2. \emph{Model Driven Engeneering}} % Change X to a consecutive number; this is for the header on each page - perhaps a shortened title

%----------------------------------------------------------------------------------------
%	SECTION 1
%----------------------------------------------------------------------------------------

\section{Model Driven Engeneering}

Model Driven Engineering (MDE)\cite{France2007} thrives to raise the level of
abstraction in program specification and increase automation in program development.
Model Driven Engineering has been a part of computer science since the very
beginning, where models has been used to raise the level of abstraction for both
problem specification for a program application or by describing parts of a
computer system.
Model Driven Engineering has been part of computer sience Model Driven
Engineering (MDE)\cite{France2007} thrives to raise the level of abstraction in
program specification and increase automation in program development. The main
idea in MDE is to use models at different levels of abstraction when developing
applications. This leads to a higher level of abstraction in program and
problem specification. This level of abstraction is obtained either through
extensive use of models to describe some design patterns in a software
application or through use of standardized models. The first option is probably
an element of MDE that is most common among software engineers. That you
implement some aspect of a system based on a model. Unified Modelling
Language\cite{UML} is an example of a modelling language often used to describe
system design patterns in an application domain. For example 

Model Driven Engineering (MDE)\cite{France2007} thrives to raise the
level of abstraction in program specification and increase automation in program
development. The main idea in MDE is to use models at different levels of
abstraction when developing applications. This leads to a higher level
of abstraction in application code and problem specification. This level of
abstraction is obtained either through extensive use of models to describe some
design patterns in a software application or through use of standardized
models. The first option is probably an element of MDE that is most common
among software engineers. That you implement some aspect of a system based on a
model. Unified Modelling Language\cite{UML} is an example of a modelling
language often used to describe system design patterns in an application
domain. The second principle of Model Driven Engineering is to increase
automation in program development, and to obtain this we use model
transformations. \\

%-----------------------------------
%	SUBSECTION 1
%-----------------------------------
\section{Metamodelling}

 A metamodel is a description of a
modelling language, where it defines elements that are used in the model.

%-----------------------------------
%	SUBSECTION 2
%-----------------------------------

\subsection{Subsection 2}


%----------------------------------------------------------------------------------------
%	SECTION 2
%----------------------------------------------------------------------------------------

\section{Main Section 2}

