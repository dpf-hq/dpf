% Chapter Template

\chapter{Background} % Main chapter title

\label{Chapter2} % Change X to a consecutive number; for referencing this
% chapter elsewhere, use \ref{ChapterX}

\lhead{Chapter 2. \emph{Background}} % Change X to a consecutive number; this is
% for the header on each page - perhaps a shortened title

%----------------------------------------------------------------------------------------
%	SECTION 1
%----------------------------------------------------------------------------------------

\section{Model Driven Engineering}
	
When a new software is created it has always been a goal to produce high
quality code at the lowest cost. To plan a software development phase from its
initial start to delivering a finished product can seem like an impossible
thing to do. Because a software development cycle rarely goes as initially
planned. Changes do occur, both in delivering high quality code and keeping the
costs down. Traditionally when model driven engineering (MDE) is used, people
think about models, for example activity diagrams and class diagrams from the
popular modeling language, UML. Where models is used to raise the level
of abstraction for the problem specification and describing how a software
should be implemented. For these software development processes models are
indirectly used in the creation of software. This means that models are
primarily used as a reference when implementing an application.

A model is an abstraction of a system, and has its origin from Latin,
\textit{modulus} that means measure or standard. A model can either be used to
represent a system before it is created or to describe some major aspects of a
system or a concept. When we hear the term model, many will think that it
is a miniature that consists of a set of nodes and arrows. But it is important
to consider that a model can also be represented by text. 

If we consider traditionally software development processes, then models are
primarily used in application requirements and use-case diagrams specify what 
the costumer wants. Then developers can specify models to detect important
functionality of the application. The software developer may for example create
flow charts, sequence diagrams, activity diagrams, class diagrams, etc, to
describe how the system should be implemented. A model for system architecture
can also be initialized for developers to handle design choices. Rational
Unified Process\cite{Rational1998} (RUP) is an example of a software
development process that is build around extensive use of models in their
initial planning phase. RUP was initially created by Rational Software
Corporation\cite{IBMRational} in 2006 and was later acquired by International
Business Machines Corporation\cite{IBM}. This is an iterative software
development process and the purpose of RUP is to be an adaptable process
framework where the software project teams decide the elements that are
required for a development cycle. Figure~\ref{fig:RUP} explains the four
different phases, Inception, Elaboration, Construction and Transition, with
different iterations for each phase that RUP provides. The Inception phase and
the Elaboration phase is the two phases where some of the example models above
are created, both under business modeling and requirements. For the Inception
phase of RUP, the idea is to create the software application without writing
any source code. This phase is  concerned with writing text and creating models
that gives the developers a detailed specification on how the program should be
implemented. In the Elaboration phase a prototype might be implemented to show
the customer a possible implementation, but this phase also consist of creating
and modifying an extensive amount text and models that specifies analysis and
design choices. The Elaboration phase focus on designing the architecture for
the software application. The goal for these two phases is to define a solid
foundation of the application before starting with the implementation and
testing. In the Construction phase the developers should know exactly how the
application should be implemented by referring to documents and models created
in earlier phases. RUP is only one example of how a software development
process could be applied to a project. Agile development processes has become
popular the last couple of years, where processes like Scrum\cite{Schwaber2001}
and Extreme Programming\cite{Beck1999} (XP) has been integrated in software
development teams all over the world. Both Scrum and XP thrives to focus more
on the implementation and on delivering high quality code than creating
documents and models. But models will always be a tool for developers, also in
agile development processes, when some aspects of a system needs to be
explained. Because to explain parts of an implementation with a model will help
to make the explanation less complex and more abstract. 

\begin{figure}[H]
	\centering
	\includegraphics[scale=0.7]{./Figures/RUP.png}
	\caption[Rational Unified Process]
	{Iterative development cycle of Rational Unified Process.}
	\label{fig:RUP}
\end{figure}

Now we have acknowledged some development processes that are commonly used in
the industry for creating software applications. Model Driven Engineering is a
software development methodology which focuses on creating and exploiting
models. And by using these models,  MDE aims at improving productivity and
quality in software development. This is achieved by not only to use models as
documentation, but instead to use models as the major artifact in a software
development cycle. The idea is to use models at different levels of abstraction
and apply model transformations to automate the implementation on these
models. This will raise the level of abstraction in program and problem
specification. We can divide these models into two main model classes, namely
development models and runtime models. Development models is used as an
abstraction above code level. These models could represent software
requirements, work flow, architecture and software implementation. These
development models are most typically used in software development process that
we described above as a supplement in developing an application. Runtime models
represents executable systems of a software application. Example of such
executable systems could be database operations or computations of data. There
has been an increase in MDE researchers that explore how runtime models can be
used to support dynamic adaptation of systems for a software
application\cite{France2007}. The idea for MDE is to be able to specify these
development models as runtime models and evolve software applications with the
use of runtime models and development models as major artifacts of a development
process. 

A typical Model Driven Software Engineering (MDSE) approach is to obtain the
running software application through model transformations that produce a more
and more detailed version of the application until an executable version is
created. We reach this level of automation by applying model transformations to
models at earlier levels of abstraction and producing models that contains a
more and more detailed description of the software. This highlights one main
advantage of a model driven approach, and that is to bridge the communication
gap between requirements/analysis and implementation\cite{Brown2008}. For a
traditional development process today there is a gap in communication between
software developers and a customer. Because a customer is usually not an expert
in designing and implementing a software application. A customer can provide a
set of requirements for a software application and take part in analysing these
requirements to make sure that the development team shares the customers thought
of the program. The requirements and analysis can be specified down to every
detail, however a software application might experience different design choices
that leads to a different implementation of the application compared what a
business needs. If the visions of MDE is adopted to a software development
process then this could help narrow gap in communication between developers and
stakeholders. Because now we can apply model transformation that changes input
models to target models that represents the design of a software application.
From these design models we can then generate implementation code through the use
of model transformations. We will describe model transformations and their
purpose in model driven engineering in more depths in chapter~\ref{Chapter3}.
To use models at each level of abstraction is less complex than implementation
code. This represents another benefit of adopting MDSE into the development
process. Because models captures and organize the understanding of a system that
results in a more clear discussions among team members and new team members. 
One approach that introduces modeling at different level of abstraction for
including MDSE in a software development process is the Model Driven
Architecture.

\subsection{Model Driven Architecture}
\label{MDA}

Model Driven Architecture (MDA) is an industry architecture developed by the
Object Management Group (OMG) and address an application development cycle. MDA
is a proposal for applying the practices of MDE to a system development. This
architecture is a good example to use when we are discussing concepts of MDE,
because of its similarity to a traditional software development process. Since
it has support for standard phases in a software development process such as
analysis, design and implementation. MDA are adopted as a reference framework to
the concept of MDA by many organizations. One reason for this is the importance
of OMG in for the software industry. MDA is build around many concepts that OMG
has released, such as the OMG specifications the Unified Modeling Language (UML)
and the Meta Object Facility(MOF). 

\begin{figure}[H]
	\centering
	\includegraphics[scale=0.5]{./Figures/MDA.png}
	\caption[Software Development with MDA]
	{a) Model-Driven-Architecture and b) Traditional development process}
	\label{fig:MDA}
\end{figure}

Figure~\ref{fig:MDA} gives a representation of the development process that the
Model Driven Architecture provides and a traditional development process. Both
of the approaches have similar starting phases, where a customer presents a
list of requirements for a software application. Now is where a MDE approach to
software development is different. Because for MDA the idea is to use models
instead of text and diagrams for the analysis and design phase. For a
traditional development process these phases usually consist of creating
diagrams that describes parts of the application. In MDA these diagrams or
models are the main artifact for the corresponding phases, instead of just a
reference for developers to use when implementing code. The architecture then
generates implementation code based on these models. A traditional software
development process would have iterations for the implementation and testing to
make sure that the application meets the demands of the customer. This process
is continued for every iteration, where developers continually use the text and
diagrams that was created earlier in the process. The idea for an MDA
development process is to provide automation between models created at each
development phase. Instead of going back to the code and to corrections and
modifications on the application a model driven software development process
goes back to analysing the problem and modify the models accordingly. With the
power of automatically changing models from one phase to another and generate
implementation code from the models at the last level of abstraction.

Figure~\ref{fig:MDA_PLATFORM} provides a representation of the models at the
different layer of abstractions that is part of the Model Driven Architecture. 

\textbf{Computation-Independent Model (CIM)} is the most abstract level of
modeling and is often referred to as a business model or domain model. The
model does not contain any computational implications to how the software
application should behave, but express exactly what the final application
should do. This model remains independent to how a system will be or currently
is implemented and represents the requirements and purpose of the system. A
Computation-Independent Model is often described by using a natural language to
define the requirements for a software application. 

\begin{figure}[H]
	\centering
	\includegraphics[scale=0.7]{./Figures/MDA_Platforms.png}
	\caption[The three levels of modeling abstraction for MDA]
	{The three levels of modeling abstraction that MDA provides.}
	\label{fig:MDA_PLATFORM}
\end{figure}

\textbf{Platform-Independent Model (PIM)} is the level of abstraction that
describes the behavior and structure of a software application. This model is
platform independent, and that means that a implementation platform is not
specified. This means that the technological platform used to implement the
software application is not defined. A Platform-Independent Model will only
address tasks that a software application can perform. These tasks are part of
the context of the business model at the top level of abstraction.

\textbf{Platform-Specific Model (PSM)} is the level of abstraction that contains
all required information for the behavior and structure of a software
application that is linked to a specific technological platform. These specific
platform technologies can be a specific programming language like a general
purpose programming language, a specific operation system or a specific database
technology. The Platform-Specific Model contains all the information that is
required for an actual implementation of the application. 

In MDA, the core activity is the starting phase, which is the way analysis is
conducted. Requirements are firstly defined and modeled as a CIM or a PIM. The
CIM and PIM provides the solution for the requirements at a very high level of
abstraction. At the computation independent level of abstraction we provide the
requirements of a solution without thinking about the actual implementation of
an application. A CIM could not only specify how an application should behave to
different situations, but could also specify how the end users utilizes the
application. For example the model could define the requirements for a web
application that provides a collection of goods that the end users can purchase.
These requirements could specify how an employee performs tasks when a new order
arrives. For a solution to an application not all of these requirements are
necessary for the implementation. The meaning of MDA is that models created at
CIM level provides the highest level of abstraction and therefore should be
readable for everyone. In figure~\ref{fig:MDA_PLATFORM} model MDA suggest
that new models are created accordingly from a set of mappings. Models that is
provided at the platform independent abstraction is not concerned with
technologies that should be used for the actual implementation. PSM is more
concerned with describing what tasks an application should perform. But tasks
that an employee should perform, like for example making a shipment ready for
transportation is not defined in a PIM. A platform specific model specifies what
implementation platform and a set of precise descriptions of the technical
details of the corresponding implementation platform. Mapping a model to another
model is essential for applying MDA to a development process. A mapping defines
correspondences between elements of two different models and can be defined
between all different models. These three models with mapping between them makes
for excellent design choices to create an environment that specifies modeling
languages. 

%-----------------------------------
%	SUBSECTION 1
%-----------------------------------


\section{Modeling Languages}

A Modeling language is defined through three core concepts. Regardless if its
either a Domain Specific Modeling Language (DSML) or a General Purpose Modeling
Language (GPML). Figure~\ref{fig:modeling_language} represents the three main
ingredients for a modeling language.

\begin{figure}[H]
	\centering
	\includegraphics[scale=0.7]{./Figures/modeling_language.png}
	\caption[Main ingredients of a modeling lanugage.]
	{The three main ingredients of a modeling language.}
	\label{fig:modeling_language}
\end{figure}

A modeling language has an abstract and a concrete syntax. The abstract syntax
describes the structure of the modeling language and how modeling elements can
be combined together. The concrete syntax on the other hand describes a specific
representation of the abstract syntax, and can be either be a graphical or
textual representation. The semantics of a modeling language describes the
meaning of these modeling elements and the different ways to combine them for
the abstract syntax and indirectly the concrete syntax. We mentioned DSML and
GPML, where these two modeling languages represents one of the main
classification of modeling languages. A modeling language can either be
classified as a domain specific or a general purpose language. DSMLs are
modeling languages that are designed for a specific domain or a concept. While
GPMLs are modeling languages that is applicable for several different domains.
A general purpose language lacks features that are special for a particular
domain. This is one of the strengths for DSLs that is created especially for a
certain domain, and therefore provide more details to a specific domain compared
to a general purpose language. The Unified Modeling Language\cite{UML,UML_SPEC}
(UML) is an example of a standardized general-purpose modeling language that
was accepted in 2000 by the International Organization for Standardization
(ISO) as an industry standard for modeling software systems. UML was initially
developed by Grady Booch, Ivar Jacobsen and James Rumbaugh at Rational Software
in the 1990s. It was later adopted by the Object Management Group in 1997 and
has since this day been continually developed by the organisation. UML is
often called a general purpose language because it is often referred to as a
suite of languages, since it provides developers and designers with the possibility to
specify applications through several different modeling languages, or diagram
types that UML often is associated with. However, in the book, ``Model-Driven
Software Engineering in Practice'' published by Marco
Brambilla, Jordi Cabot and Manuel Wimmer in 2012, they state the following.
\textit{If we think to the general modeling problem, we can see UML as
a DSL tailored to the specification of (mainly object-oriented) software
systems\cite{Brambilla:MDSE}.} This means that to decide whether UML is a DSL or
a GPL is not a binary choice. But we mostly see UML as a general purpose
modeling language, since it offers a wide variety of modeling languages that
designers and developers can use to specify system abstractions. Whether a
modeling language is classified as a general purpose or a domain specific
modeling language it requires that it is described by an abstract syntax. Both
the abstract syntax and the concrete syntax of a modeling language is
represented as models. Therefore the specification of the abstract syntax is
often referred to as a meta-model. 

\subsection{Meta-modeling}

Models are a major artifact in the concept of model driven engineering (MDE). It
is essential to look at every model as instances of some more abstract model.
And therefore we can define a meta-model as yet another abstraction that
highlights properties of an instance model. Meta-modeling represents a vital
part of MDE and constitutes the definition of a modeling language. A meta-model
defines the abstract syntax and provides a description of a modeling language.
Another popular definition for describing a meta-modeling is that it is a
``model of models". This definition is both unhelpful and incorrect according
to Steve Cook and Stuart Kent in their paper\cite{Cook2008} published in 2008.
They think that a better definition for a meta-model is that ``it is a model of
the concepts expressed by a modeling language.'' The exact definition of a
meta-model is highly debated amongst MDE researchers\cite{Rutle_thesis}.

\begin{figure}[H]
	\centering
	\includegraphics[scale=0.6]{./Figures/SimpleMetamodel.png}
	\caption[Example of a model and meta-model]
	{A simple example of a model and its meta-model.}
	\label{fig:SimpleMeta-model}
\end{figure}

Figure~\ref{fig:SimpleMeta-model} shows a simple example of an instance model
and its corresponding meta-model. This model has two classes, Student and Course, and
a bidirectional association, take course and has students, that relate
these two classes. The model is specified by a meta-model that consists of two
meta-classes Class and Association and an association between them. Both Student
and Course are an instance of the meta-class Class, while the association
between Student and Course are instance of the meta-class Association. The
modeling language that describes this model corresponds to the Unified Modelling
Language.

\subsubsection*{Meta-Object Facility}

The Meta-Object Facility\cite{MOF} (MOF) is an Object Management Group standard
for model driven engineering. The Object Management Group was in need of a
architecture to define the UML. Therefore the Meta-Object Facility has its
origin from UML. Through this process of finding a common platform for UML, OMG
designed a four layered architecture that provides a semi-formal approach to
creating meta-models. MOF became a language for defining abstract syntax for
modeling languages.  

\begin{figure}[H]
	\centering
	\includegraphics[scale=0.6]{./Figures/MOFLayers.png}
	\caption[Meta Object Facility]
	{Example of Meta Object Facility and its four layers.}
	\label{fig:MOFLayers}
\end{figure}

Figure~\ref{fig:MOFLayers} gives a impression of the four layers that are
available in the Meta-Object Facility. At the top level, M\textsubscript{3}
there is a meta-meta-model called MOF. This meta-meta-model is meant to both
describe it self and conform to itself. This MOF meta-meta-model is then used to
describe meta-models at the M\textsubscript{2} level. The UML meta-model is an
example of such a meta-model. The idea is that these meta-models are specified by
some meta-modeling language. These meta-models at the M\textsubscript{2} layer
describes elements that can be used in the M\textsubscript{1} layer, namely
models. These could be models specified by the modeling language, UML. Finally
at the M\textsubscript{0} we have an instance model of a real world object. 



%-----------------------------------
%	SUBSECTION 2
%-----------------------------------



%----------------------------------------------------------------------------------------
%	SECTION 2
%----------------------------------------------------------------------------------------
 
\subsection{Constraints}

A constraint is a Boolean condition for some modeling element. These modeling
elements can have constraints defined on objects, classes, attributes,
links, associations, etc. A constraint is a restriction on how these
elements should behave. Constraints on elements such as those above can be
expressed with a natural language or by a formal language, such as the
Object Constraint Language\cite{OCL} (OCL). The Object Constraint Language (OCL)
is a declarative programming language for describing constraints that applies to
UML models. Before UML became an adopted standard of the Object Management Group
(OMG), OCL was an extension language to UML. Now OCL can be used with any
Meta-Object Facility (MOF) meta-model, including UML. A software developer
can in combination with UML and OCL specify models. OCL is used as a supplement
to UML, and therefore would refer a to non-existing model element without UML
diagrams\cite{Warmer:2003:OCL:861416}.

The difference between object and classes needs to be specified. A class
is often a meta element for an object. This means that a class could be part of
a model that describes an object element, and therefore an object element is
typed by class element\cite{OO_UML}. Figure~\ref{fig:SimpleMeta-model} describes
the two object elements Student and Course that are an instance of the
meta-element Class.

\begin{figure}[H]
	\centering
	\includegraphics[scale=0.7]{./Figures/Constraints.png}
	\caption[Simple model with constraints]
	{Example of a simple model with attached and structural constraints.}
	\label{fig:Constraints}
\end{figure}

These restrictions on model elements can either be a structural constraint or an
attached constraint. These structural constraints are defined in the structure
of the models. In figure~\ref{fig:Constraints} we have extended the model we
introduced in figure~\ref{fig:SimpleMeta-model} with some  modeling elements. We
have created an association that specifies that a student can date other
students. In this model we can see that the model has three multiplicity
constraints that are part of the models structure. A multiplicity constraint for
an association restricts the number of objects that are related to a given
object. From the association constraint on this model we can see that a student
requires to take at least one course and up to a maximum of six courses. The
models restricts a student to not participate in any courses. The second
structural constraint requires a course to have at least five students for
this course to be part of a semester, and the course can have an arbitrary
maximum number of students participating this course. The association dates,
between two students for this model has an attached constraint, that is
specified in the declarative language OCL. The general form of an attached
constraint has a context, in this case a Student, that specifies what object the
constraint includes. The is a constraint name followed by a Boolean
expression. The attached constraint has a name \textit{``Irreflexive''} followed
by a Boolean OCL expression that explicitly refers to itself. This constraint
specifies that a student is unable to date her or him self. Constraints has a
vital part in model driven engineering to measure the quality and precision of a
model. A model without constraints does not work in practice. In
\textit{The Object Constraint Language: Getting Your Models Ready for
MDA}\cite{Warmer:2003:OCL:861416}, Jos Warmer and Anneke Kleppe states that a
model without constraints would be severely underspecified. Constraints
expressions written in OCL are unambiguous and results in a more precise and
detailed model. If we where to remove both the structural and attached
constraints from figure~\ref{fig:Constraints} then the model is less
informative. There is no understanding on how the objects are related to
one another. 

\section{Language Workbenches}

Language workbenches are tools that lets user specify
their own Domain Specific Language (DSL) and include editing tools for the newly
created language. A workbench should consist of Integrated Development
Environment (IDE) that lets users create their own DSMLs. The concepts behind a
language workbench is that the tool does not just provide the users with an IDE
to create DSLs, but also generates a new IDE where this newly created DSL can
be edited. In addition to an IDE that provides creation and editing of a newly
created language a workbench should define support for code generation, model
transformation, model versioning, etc\cite{Lamo2013}.
Figure~\ref{fig:langauge_workbench} is provided in the paper, \textit{``DPF
Workbench: a multi-level language workbench for MDE"}, that was published by
Yngve Lamo, Xiaoliang Wang, Florian Mantz, �yvind Bech, Anders Sandven and
Adrian Rutle in 2013. The figure presents the intended use of language
workbenches and consist of two phases. The first handles the definition of a new
DSML and the creation of tool support such as code generation, editors, model
transformations, etc. A language workbench is created by a domain expert in
collaboration with an experienced developer. The latter describes the actual
usage of this newly created workbench, where developers can utilize the DSML
environment to create models, generate implementation code, etc. Language
workbenches are a very young field in computer science, and there are many
existing solutions that is open for the public to use. These concepts have the
potential to change the face of programming as we know
it\cite{fowler2010domain}, but the concepts of workbenches are still fresh to 
computer science. 

\begin{figure}[H]
	\centering
	\includegraphics[scale=0.6]{./Figures/langauge_workbehcn.png}
	\caption[Intended use of language workbenches]
	{Intended use of language workbenches.}
	\label{fig:langauge_workbench}
\end{figure}

\begin{figure}[H]
	\centering
	\includegraphics[scale=0.6]{./Figures/EMF_Diagram_Picture.png}
	\caption[Ecore model represented by Ecore Tools]
	{A graphical representation of an Ecore model.}
	\label{fig:EMF_Diagram}
\end{figure}

\begin{figure}[H]
	\centering
	\includegraphics[scale=0.8]{./Figures/EMF_Ecore_Picture.png}
	\caption[Tree based representation of an Ecore model]
	{Tree based representation of an Ecore model.}
	\label{fig:EMF_Ecore}
\end{figure}

\begin{figure}[H]
	\centering
	\includegraphics[scale=0.7]{./Figures/EMF_GenModel.png}
	\caption[Representation of an EMF Generator model]
	{Representation of a generated EMF Generator model.}
	\label{fig:EMF_Ecore}
\end{figure}

\subsection{EMF}

Eclipse Modeling Framework is originally based on Meta Object Facility (MOF)
provided by the Object Management Group (OMG). In 2003 EMF designers contributed
to designing the MOF 2.0 version of the standard and was named Essential EMOF
(EMOF). EMF use EMOF and have designed a meta-model called Ecore and is the
meta-model that EMF use to specify new DSLs and create a tooling environment. 


\subsection{Visual Modeling and Transformation System}

\begin{figure}[H]
	\centering
	\includegraphics[scale=0.5]{./Figures/VMTS}
	\caption[Creating domain specific modeling language in VMTS]
	{Specifying a domain specific modeling language in VMTS.}
	\label{fig:VMTS}
\end{figure}

\section{Diagram Predicate Framework}

Diagram Predicate Framework\cite{Rutle_thesis,Rossini_thesis,Lamo2013} (DPF) is
an ongoing research project that was first initiated by Bergen University
Collage and the University of Bergen in Norway 2006. With features likes
meta-modeling, model transformation and model management, DPF aims at
formalising concepts of model-driven engineering. DPF is based on category
theory and graph transformations and is an extension of the Generalised
Sketches\cite{Diskin2003} formalism that was initially developed by Zinovy
Diskin.

In October 2002 Dominique Duval published a paper where he specified that a
specification can be considered as a directed graph with additional structure
in the same way that a theory can be considered as a category with additional
structures\cite{Duval2003}. Generalised Sketches by Zinovy Diskin utilize the
concept of sketches. A sketch, first introduced by Ehresman in 1966, is a
directed graph that provides additional properties, such as colimit, limit and
constraints. DPF utilize this concept through an diagrammatic approach to
meta-modeling and to facilitate the concepts of MDE. The framework provides the
possibility to define an unlimited layers of meta-modeling. In DPF models are
represented as specifications. 

\begin{itemize}
  
\item A \emph{specification $\spec{S}$} = (S,
C\textsuperscript{$\spec{S}$}:$\Sigma$) consist of an underlying graph S and a
set of atomic constraints C\textsuperscript{$\spec{S}$}. 

\item Atomic constraints are specified by predicates, that is a Boolean value
function, from a predefined signature $\Sigma$.

\item A signature $\Sigma$ = ($\Pi$ \hspace{1 mm}, \hspace{1 mm}$\alpha$)
consist of a collection of predicates. 
  
\end{itemize}

A \emph{specification $\spec{S}$} has an underlying graph S that contains
modeling elements that defines the model structure of the specification. These
modeling elements are always represented as a node and an arrow. However these
nodes and arrows could be specified through several underlying layers of
meta-models or specifications. The \emph{specification $\spec{S}$} also consist
of a set of constraints, these constraints will restrict the model structure of
a new instance model of this specification. Figure~\ref{fig:DPF_Spec} presents a
specification $\spec{S}$\textsubscript{2}, that is defined by an underlying
specification $\spec{S}$\textsubscript{3} and describes a modeling language for
some $\spec{S}$\textsubscript{1} specification. This specification includes two
nodes Condition and Activity, two arrows ChoiceOut and Message and  two sets of
atomic constraints. The first constraint defines that a Condition element has
to be connected to exactly one Activity element for this structure. The second
constraints specifies for this graph structure that an Activity element cannot
be associated with it self. These constraints examples are specified as a
collection of predicates from a predefined signature $\Sigma$. The table in
figure~\ref{fig:DPF_Spec} represent some of the predicates from this
collection. A predicate is represented by an unique symbol $\Pi$, a shape graph
$\alpha$, a proposed visualisation and a semantic interpretation

\begin{figure}[H]
	\centering
	\includegraphics[scale=0.7]{./Figures/DPF_spec_constraints.png}
	\caption[A specification and some predefined diagrammatic predicate attached]
	{A specification $\spec{S}$\textsubscript{2} with some attached predicates.}
	\label{fig:DPF_Spec}
\end{figure}

An instance specification $\spec{S}$\textsubscript{n} that is initialised from
a specification $\spec{S}$\textsubscript{n+1} defines a graph homomorphism
between two underlying graphs. There is a graph homomorphism,
T$\longrightarrow$S, between the underlying graph T of
$\spec{S}$\textsubscript{n} and the underlying graph S of
$\spec{S}$\textsubscript{n+1}\cite{Lamo2013}. This graph homomorphism
T$\longrightarrow$S must satisfy the set of atomic constraints,
C\textsuperscript{$\spec{S}$} from the specification
$\spec{S}$\textsubscript{n+1}. The nodes and arrows that are defined in the
underlying graph S from a specification $\spec{S}$\textsubscript{n+1} describes
the modeling elements that a specification $\spec{S}$\textsubscript{n} can use
to create a graph structure.

\cite{Brambilla:MDSE}

\begin{figure}[H]
	\centering
	\includegraphics[scale=0.7]{./Figures/MOF_vs_DPF}
	\caption[OMG's layers of meta-modeling and multilayer modeling]
	{OMG's layers of meta-modeling and an arbitrary layer of meta-modeling.}
	\label{fig:MOF_vs_DPF}
\end{figure}



 
\section{DPF Editor}

 
\begin{table}[ht]
\renewcommand*\arraystretch{1.5}
\centering
\begin{tabular}{| c | c | c | c | c | c |}
\hline
Tool & Layers & Dia.   & Const.   & Platform & GUI\\
	 & 		  & Const. & Language &          & \\
\hline
EMF/GMF & 2 & & OCL, EVL, & Java VM & $\surd$ \\
		&  	& & Java      &         & \\
\hline
VMTS    & $\infty$ & & OCL & Windows & $\surd$ \\
\hline
AToM\textsuperscript{3} & 2 & & OCL, Python & Python, Tk/tcl & $\surd$ \\
\hline
GME  & 2 & & OCL & Windows & $\surd$ \\
\hline
metaDepth & $\infty$ & & EVL & Java VM & \\
\hline
DPF & $\infty$ & $\surd$ & Predefined & Java VM & $\surd$ \\
	& 		   &         & validator  &         & \\
\hline

\end{tabular}
\caption{Comparing model transformation tools.}
\end{table} 