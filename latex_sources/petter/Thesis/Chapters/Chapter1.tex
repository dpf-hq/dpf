\chapter{Introduction} % Main chapter title

\label{introduction} % Change X to a consecutive number; for referencing this
% chapter elsewhere, use \ref{ChapterX}

\lhead{Chapter 1. \emph{Motivation}} % Change X to a consecutive number; this
% is for the header on each page - perhaps a shortened title
\singlespacing

\section{Motivation}

Model Driven Engineering (MDE) has been around for quite some time now and is by
no references new to computer science. The use of miniatures or a visual
representation of a system to provide a explanation of some problem has always
been around. To be able to draw an abstraction of a complex system or problem
on a piece of paper to supplement a explanation often tends to make the explanation
easier to understand. The explanation does not necessary have to be difficult
to understand, but a miniature or a model will make the explanation easier to
understand. This was how i experienced the use of models during my bachelor
period. That they were meant to represent abstractions of systems and programs. 

When i started to work on my master program we started to visit more concepts
around MDE and how we can use models and model transformations to automate a
software application process. The first tools that supported this vision of MDE
were the Computer-Aided Software Engineering (CASE) and was developed in the
1980s. Tools like CASE thrived to achieve this vision of MDE to be a fully
usable approach to software development. This meant to use models as the major
artifact in a software application. Where in the initially phases of a
software development models would provide an abstraction of the problem and
evolve with more details through the development process. These models would
evolve to more specific technology based abstractions with the help of model
transformations and in the end fully executable software implementation of the
application would be generated.

As the years went by MDE has become a strong foundation to create domain
specific modeling languages. With the possibility to define meta-models with
constraints proved to be a viable solution to create the structure of a
modeling language or a specific modeling language. Based on this the Object
Management Group (OMG) created the Unified Modeling Language\cite{UML_SPEC}
(UML) that became a standard for creating modeling languages. Many modeling
tools adapts UML in the process of creating a domain specific modeling
environment.

Lately a diagrammatic approach to utilize the visions of MDE has become more
popular amongst MDE researchers. Where the focus is more on graphs and graph
theory. The Diagram Predicate Framework (DPF) is such a framework that take
advantage of category theory and graph transformations to provide a formal
approach to meta-modeling, model transformation and model management. For this
thesis we have the following research question, \textit{Can we extend the DPF
Workbench with an editor that support model to model transformations between
different Domain Specific Modeling Languages.}

Prototyping Methods
This subsection covers methods involving software
prototyping and is subdivided into prototyping styles,
targets, and evaluation techniques.
Prototyping styles.  ) The
prototyping styles topic identifies the various
approaches: throwaway, evolutionary, and executable
specification.
Prototyping target. Examples of the
targets of a prototyping method may be requirements,
architectural design, or the user interface.
Prototyping evaluation techniques. This topic covers
the ways in which the results of a prototype exercise
are used.

\section{The structure of the Thesis}

This thesis is structured with the following sections.

\begin{description}
  \item[Chapter 2.] This chapter is meant to explain background material to this
  thesis. Where we discuss the visions of model driven engineering. We also
  consider a specific design approach to these visions. Then we discuss modeling
  languages and their role in language workbenches. At the end of the chapter we
  go into detail on the DPF environment. 
  
  \item[Chapter 3.] This chapter introduces model transformation in general. We
  discuss the basic concepts of model transformations and how they are used in
  MDE. Later in the chapter we try to classify model to model transformations
  and explain design choices behind the graph based approach to achieve this.
  
  
  \item[Chapter 4.] In chapter 4 we describe the problem at hand and how we want
  to approach this. We also look at some specific model transformation tools and
  compare design choices. 
  
  
  \item[Chapter 5.] This chapter describes how we created a model
  to model transformation environment for the framework. We describe the model
  transformation environment we integrated with DPF and how this works with the
  transformation editor. 
  
  
  \item[Chapter 6.] Here we want evaluate our solution and also discuss
  functionality that should be included in future versions of the tool. Then we
  want to compare our solution with existing transformation tools and provide a
  conclusion at the end. 
\end{description}