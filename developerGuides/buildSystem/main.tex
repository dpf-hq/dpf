\documentclass[a4paper]{article}

\usepackage[utf8]{inputenc}
\usepackage{listings}
\usepackage{url}
\usepackage{graphicx}
\usepackage{authblk}
\usepackage{xspace}
\usepackage{tabularx}
\usepackage{verbatim}


\title{Building DPF releases}


\begin{document}

\maketitle

\section{Tycho and the DPF build system}

The DPF project uses a build system based on Maven and Tycho plugin. This allows us to build, test and package all the plugins belonging to the DPF
project using a single command. Furthermore, using a well supported build system allows the implementation of continuous integration solutions.
In this document I will treat Mavn as a commend line tool, however it is also possiblities to integrate Maven with Eclipse.

\subsection{Installing Dependencies}

The only technology required to build DPF is Maven. Maven will, based on the pom.xml files in the DPF plugins, automatically download
any other requirements. Instructions on how to install and use maven are availbale at \url{http://maven.apache.org/guides/getting-started/maven-in-five-minutes.html}. 

\subsection{Using maven with DPF}

Maven is controlled by pom.xml files. Each plugin as well as the root directory for the projects have one pom.xml file each.

\begin{itemize}
\item Why a build system
\item Tycho and maven
\item Intalling maven
\item Uses of the build system
\end{itemize}


\section{How build a release}

\begin{itemize}
\item Running tests
\item Running builds
\item Building fot other platforms
\item The DPF Eclipse application
\end{itemize}

\section{Adding a new plugin to the build}

\begin{itemize}
 \item creating a new pom.xml
 \item Adding the new plugin to the project build
\end{itemize}


\end{document}
